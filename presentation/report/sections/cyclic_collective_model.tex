\subsection{Model}
For one single blade, the infinitesimal lift force applied to the airfoil is given by:
\begin{align*}
    dF_l=\frac{1}{2} C_l c \rho v^2 \cdot dr,
\end{align*}
where $c$ is the chord of the blade (assumed to be constant over the length of it) and $C_l$ is the lift coefficient. Keep in mind that in this section the value of the latter depends on the angle of attack of the blade, which is assumed to be independent of the distance from the rotor hub and equal to:
\begin{align*}
    AoA=-\alpha sin\psi + \beta cos\psi,
\end{align*}
where $\psi$ is the angle the blade has traveled rotating from the positive direction of the $x$ axis of the body frame, while $\alpha$ and $\beta$ become the angles that the swashplate forms with $x$ and $y$ axes. \\
Moreover, $v$ is the relative air flow velocity, namely the sum of the tangential speed of the blade and the orthogonal component to the blade of the speed of the helicopter (assuming to have no wind contribution): 
\begin{gather*}
    v=\omega r+V_\perp \quad \text{with} \\
    \begin{split}
        V_\perp &= V^b_x cos\left(\frac{\pi}{2}-\psi \right)+V^b_y cos(\pi-\psi) \\
        &= V^b_x sin\psi-V^b_y cos\psi.
    \end{split}
\end{gather*}
\\
This results in the following expression for infinitesimal lift force:
\begin{align*}
    dF_l=\frac{1}{2} C_l c \rho (V_\perp^2+\omega^2 r^2+2\omega r V_\perp) \cdot dr.
\end{align*}
By integrating over the length of the blade we get the total lift force for a single blade:
\begin{align*}
    F_l=\frac{1}{2} C_l c \rho  \left(V_\perp^2 r+\omega^2 \frac{r^3}{3}+2\omega V_\perp\frac{r^2}{2}\right)
\end{align*}

The very same process is carried out for the drag force of each blade, yielding:
\begin{align*}
    F_d = \frac{1}{2} C_d c \rho \left(V_\perp^2 l+\omega^2 \frac{l^3}{3}+2\omega V_\perp\frac{l^2}{2}\right)
\end{align*}

Finally, the module of the force generated by one blade is given by the difference between the lift and drag forces:
\begin{align*}
    F_i=F_{l,i}-F_{d,i}, \quad i=1,2,
\end{align*}
and logically the norm of the total force for one of the two rotors is:
\begin{align*}
    F=F_1+F_2.
\end{align*}