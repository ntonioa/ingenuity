\section{Modeling the thrust vectors}
In the first level of abstraction, each of the two rotors of Ingenuity generates a thrust vector in a commanded direction, applied on the rotating hub. Not only does this result in a force capable of moving the helicopter, but it also generates a torque that can be used to control its orientation. \\The helicopter is treated as a simple rigid body without taking into account the aerodynamics of the blades, therefore the only external disturbance is the gravitational force. 

From now on, we will consider an inertial reference frame (superscript $^i$) and a non-inertial one attached to the drone body (superscript $^b$). \\ The relative orientation of the latter with respect to the former is expressed in roll, pitch and yaw angles (i.e. the three components of $W$), or alternatively through the corresponding rotation matrix (in compact trigonometric notation with $\sin(W_i) = s_i$ and $\cos(W_i) = c_i$):
\begin{align*}
    R(W) = R = \begin{bmatrix}
        c_2 c_1& s_2 s_1 c_3 - s_1 c_3 & s_3 s_2 c_1 + s_1 s_3 \\
        c_2 s_3 & s_2 s_1 s_3 + c_3 c_1 & s_3 s_2 c_1 - s_1 c_3 \\
        -s_2 & s_1 c_2 & c_2 c_1
    \end{bmatrix}
\end{align*}

% \begin{align*}
%     R = \begin{bmatrix}
%         c_\theta c_\phi& s_\theta s_\phi c_\psi - s_\phi c_\psi & s_\psi s_\theta c_\phi + s_\phi s_\psi \\
%         c_\theta s_\psi & s_\theta s_\phi s_\psi + c_\psi c_\phi & s_\psi s_\theta c_\phi - s_\phi c_\psi \\
%         -s_\theta & s_\phi c_\theta & c_\theta c_\phi
%     \end{bmatrix}
% \end{align*}
