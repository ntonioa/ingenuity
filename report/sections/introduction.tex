\section{Introduction}
A fully-actuated (or even redundant) robot is capable of performing operations and accomplishing tasks with extreme accuracy and repeatability. On the other hand, when the number of actuators is less than the number of degrees of freedom, the capabilities of the robot are reduced, and it is said to be underactuated. \\
Underactuation is not only a hurdle, but it brings a whole host of advantages, first and foremost efficiency (but also spectacularity) both in design and operation, although this comes at the cost of severely complicating their control.

A widely-used class of optimal control algorithms that allows good trajectory tracking performance of systems with a high degree of nonlinearity such as underactuated robots is that of Differential Dynamic Programming (DDP) methods. Introduced in 1966 by Mayne \cite{mayne}, they can be exploited to implement a receding-horizon control strategy reminiscent of Model Predictive Control \cite{tassa07}. The optimal control problem to be solved can also embed input constraints, which are common in robotics applications \cite{tassa14}. 

This document addresses the implementation of a controller for two among the most significative examples of underactuated robots, namely the pendubot and the acrobot. For robustness concerns, DDP is used to find optimal trajectories continuously in real-time according to a receding horizon logic, in the meantime satisfying hard constraints on the control sequence. The performance of the controller is evaluated through simulations, and then the results are discussed.