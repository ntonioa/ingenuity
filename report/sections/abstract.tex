\begin{abstract}
    Underactuated robots, characterized by having fewer actuators than degrees of freedom, present unique control challenges and advantages. Here, we investigate the theory behind Differential Dynamic Programming (DDP), exploiting the perks that it offers for optimally controlling nonlinear systems. By reviewing the existing literature, we refine the DDP control algorithm through regularization measures that enhance its reliability and performance, such as the exploitation of a Levenberg-Marquardt parameter and line search. Credibility is improved through the consideration of hard constraints on the control input and the combination of our algorithm with a receding horizon logic reminiscent of Model Predictive Control. We provide pseudo-codes that explicate the overall control strategy step-by-step. To validate our approach, we apply the algorithm to the control of the pendubot and the acrobot, two underactuated robots with similar dynamics. Finally, the paper is endowed with a discussion of the results, which demonstrate the effectiveness of our approach in controlling underactuated robots.
\end{abstract}